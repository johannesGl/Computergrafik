\documentclass[12pt]{article} 
\usepackage{german} 
\usepackage[utf8]{inputenc} 
\usepackage{latexsym} 
\usepackage{amsfonts} 
\usepackage{amsmath}
\usepackage{amssymb}
\usepackage{MnSymbol}
\usepackage[colorlinks=true,urlcolor=blue]{hyperref}
\usepackage{listings}
\usepackage{graphicx}
\usepackage[utf8]{inputenc}
\usepackage[german]{babel}
\usepackage[T1]{fontenc}
\usepackage{wasysym}
\usepackage{stmaryrd}
\usepackage{makeidx}
\usepackage{pgfplots}
\usepackage{listings}
\usepackage[ruled,vlined]{algorithm2e}
\usepackage[onehalfspacing]{setspace}
\usepackage[headsepline]{scrpage2}
\usepackage{enumerate}
\pagestyle{plain}
\pagestyle{scrheadings}
\usepackage[left=2.5cm,right=2.5cm,top=2.5cm,bottom=2.5cm]{geometry}

\lstset{language = Java, breaklines = true, showstringspaces = false}

 %Titel
\begin{document}
\begin{center}
 \textbf{\Large Computergrafik Blatt 5}\\
 \emph{Anton Zickenberg, Johannes Gleichauf}
\end{center}

\section*{Aufgabe 1}
Um zu zeigen, dass die beiden Beleuchtungsmodelle identisch oder nicht identisch sind, genügt es zu zeigen, dass die beiden Modelle gleich oder eben nicht gleich sind. Wir beginnen damit, die beiden Modelle gleichzusetzen:\\
$R_{\text{Phong}} = R_{\text{Blinn}}$\\
$E \cdot (R \cdot V) ^{\alpha} =  E \cdot (H \cdot N)^{\alpha}$\\
Da wir auf beiden Seiten denselben Faktor E und $\alpha$ haben, können wir diesen streichen und erhalten:\\
$R \cdot V = H \cdot N$\\
Dies ist die Aussage, die wir zeigen oder widerlegen wollen. Nun gilt $(R \cdot V)^{\alpha} = cos^{\alpha}\ \Omega$ oder vereinfacht $R \cdot V = cos\ \Omega$. Sei äquivalent dazu für das Blinn Beleuchtungsmodell der Winkel $\theta$ zwischen der Normalen und dem Halbwegsvektor gegeben. Für diesen Winkel gilt $cos\ \theta = H \cdot N$. Eingesetzt erhalten wir also:\\
$cos\ \Omega = cos\ \theta$\\
Betrachten wir nun Abbildung 1, so ist offensichtlich, dass $\theta < \Omega$ gilt. Daraus folgt für uns:\\
$cos\ \Omega > cos\ \theta$\\
Somit sind die beiden Beleuchtungsmodelle anhand gewählter Winkel aus Abbildung 1 nicht gleich.
\section*{Aufgabe 2}

\end{document}