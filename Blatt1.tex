\documentclass[12pt]{article} 
\usepackage{german} 
\usepackage[utf8]{inputenc} 
\usepackage{latexsym} 
\usepackage{amsfonts} 
\usepackage{amsmath}
\usepackage{amssymb}
\usepackage{MnSymbol}
\usepackage[colorlinks=true,urlcolor=blue]{hyperref}
\usepackage{listings}
\usepackage{graphicx}
\usepackage[utf8]{inputenc}
\usepackage[german]{babel}
\usepackage[T1]{fontenc}
\usepackage{wasysym}
\usepackage{stmaryrd}
\usepackage{makeidx}
\usepackage{pgfplots}
\usepackage[ruled,vlined]{algorithm2e}
\usepackage[onehalfspacing]{setspace}
\usepackage[headsepline]{scrpage2}
\usepackage{enumerate}
\pagestyle{plain}
\pagestyle{scrheadings}
\usepackage[left=2.5cm,right=2.5cm,top=2.5cm,bottom=2.5cm]{geometry}

 %Titel
\begin{document}
\begin{center}
 \textbf{\Large Computergrafik Blatt 1}\\
 \emph{Anton Zickenberg, Johannes Gleichauf}
\end{center}

\section*{Aufgabe 1}
\includegraphics*[scale=0.165]{Screenshot.png}
\section*{Aufgabe 2}
\begin{enumerate}[a)]
\item int a = 2; //Integer-Variable a\\ 
	  int *pa = $\&$a ; //Pointer auf Variable a\\ 
	  a++; //Was steht in Variable a?\\ 
	  Der Wert der Variable ist nun 3\\
	  \\	  
	  (*pa)++; //Was steht jetzt in Variable a?\\ 
	  Der Wert der Variable ist nun 4\\
	  \\
	  *pa = 0; //Welchen Wert hat Variable a?\\
	  Der Wert der Variable ist nun 0\\
	  \\ 
	  pa++; //Was bewirkt dieses Statement?\\ 
	  		//Warum ist es problematisch?\\
\item Ein Pointer zeigt auf den Wert einer Speicheradresse. Eine Referenz ($\&$-Operator) besitzt die Speicheradresse einer Variablen.
\item 
\end{enumerate}
\section*{Aufgabe 3}

\end{document}