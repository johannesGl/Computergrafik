\documentclass[12pt]{article} 
\usepackage{german} 
\usepackage[utf8]{inputenc} 
\usepackage{latexsym} 
\usepackage{amsfonts} 
\usepackage{amsmath}
\usepackage{amssymb}
\usepackage{MnSymbol}
\usepackage[colorlinks=true,urlcolor=blue]{hyperref}
\usepackage{listings}
\usepackage{graphicx}
\usepackage[utf8]{inputenc}
\usepackage[german]{babel}
\usepackage[T1]{fontenc}
\usepackage{wasysym}
\usepackage{stmaryrd}
\usepackage{makeidx}
\usepackage{pgfplots}
\usepackage{listings}
\usepackage[ruled,vlined]{algorithm2e}
\usepackage[onehalfspacing]{setspace}
\usepackage[headsepline]{scrpage2}
\usepackage{enumerate}
\pagestyle{plain}
\pagestyle{scrheadings}
\usepackage[left=2.5cm,right=2.5cm,top=2.5cm,bottom=2.5cm]{geometry}

\lstset{language = Java, breaklines = true, showstringspaces = false}

 %Titel
\begin{document}
\begin{center}
 \textbf{\Large Computergrafik Blatt 4}\\
 \emph{Anton Zickenberg, Johannes Gleichauf}
\end{center}

\section*{Aufgabe 1}
Es gilt:\\
$M_{orth} = M_{per} \cdot P$
Stellen wir diese Formel um erhalten wir:\\
$M_{orth} \cdot  M_{per}^{-1} =  P$\\
Wir beginnen damit, die Matrix $M_{per}$ zu invertieren:\\
$\left(
\begin{array}{cccc|cccc}
  \frac{2n}{r-l} &  0 &  \frac{r+l}{r-l} & 0 & 1 & 0 & 0 & 0 \\
  0 &  \frac{2n}{t-b} &  \frac{t+b}{t-b} & 0 & 0 & 1 & 0 & 0 \\
  0 & 0 & \frac{-(f+b)}{f-n} & \frac{-2fn}{f-n} & 0 & 0 & 1 & 0\\
  0 & 0 & -1 & 0 & 0 & 0 & 0 & 1
\end{array}
\right)$
\\
\\
\\
$\left(
\begin{array}{cccc|cccc}
  \frac{2n}{r+l} &  0 &  1 & 0 & \frac{r-l}{r+l} & 0 & 0 & 0 \\
  0 &  \frac{2n}{t+b} &  1 & 0 & 0 & \frac{t-b}{t+b} & 0 & 0 \\
  0 & 0 & 1 & \frac{-2fn}{-(f+n)} & 0 & 0 & \frac{f-n}{-(f+n)} & 0\\
  0 & 0 & -1 & 0 & 0 & 0 & 0 & 1
\end{array}
\right)$
\\
\\
\\
$\left(
\begin{array}{cccc|cccc}
  \frac{2n}{r+l} &  0 &  0 & 0 & \frac{r-l}{r+l} & 0 & 0 & 1 \\
  0 &  \frac{2n}{t+b} &  0 & 0 & 0 & \frac{t-b}{t+b} & 0 & 1 \\
  0 & 0 & 0 & \frac{-2fn}{-(f+n)} & 0 & 0 & \frac{f-n}{-(f+n)} & 1\\
  0 & 0 & -1 & 0 & 0 & 0 & 0 & 1
\end{array}
\right)$
\\
\\
\\
$\left(
\begin{array}{cccc|cccc}
  1 &  0 &  0 & 0 & \frac{r-l}{2n} & 0 & 0 & \frac{r+l}{2n} \\
  0 &  1 &  0 & 0 & 0 & \frac{t-b}{2n} & 0 & \frac{t+b}{2n} \\
  0 & 0 & 1 & 0 & 0 & 0 & 0 & -1\\
  0 & 0 & 0 & 1 & 0 & 0 & \frac{f-n}{-2fn} & \frac{-(f+n)}{-2fn}
\end{array}
\right)$\\
\\
Nun haben wir auf der rechten Seite unsere Inverse Matrix. Nun setzen wir diese in die obere Gleichung ein:\\
$\left(
\begin{array}{cccc}
\frac{2}{r-l} & 0 & 0 & \frac{-(r-l)}{(r-l)} \\
0 & \frac{2}{(t-b)} & 0 & \frac{-(t+b)}{(t-b)} \\
0 & 0 & \frac{-2}{(f-n)} &  \frac{-(n+f)}{(f-n)} \\
0 & 0 & 0 & 1
\end{array}
\right) \cdot \left(
\begin{array}{cccc}
\frac{r-l}{2n} & 0 & 0 & \frac{r+l}{2n} \\
 0 & \frac{t-b}{2n} & 0 & \frac{t+b}{2n} \\
0 & 0 & 0 & -1 \\
 0 & 0 & \frac{f-n}{-2fn} & \frac{-(f+n)}{-2fn} \\
\end{array}
\right) = P$\\
und erhalten:\\
$ \left(
\begin{array}{cccc}
\frac{1}{n}& 0 & -\frac{(f-n)(l-r)}{2fn(l+r)} & \frac{l+r}{n(r-l)} - \frac{(-f-n)(l-r)}{2fn(l+r)}\\
 0 & \frac{1}{n} & -\frac{(f-n)(-b-t)}{2fn(t-b)} & \frac{b+t}{n(t-b)} - \frac{(-f-n)(-b-t)}{2fn(t-b)} \\
0 & 0 & -\frac{-f-n}{2fn} & \frac{2}{f-n} - \frac{(-f-n)^2}{2f(f-n)n} \\
0 & 0 & -\frac{f-n}{2fn} & -\frac{-f-n}{2fn} \\
\end{array}
\right)$
\section*{Aufgabe 2}
Im Framework bearbeitet

\end{document}